
\documentclass[12pt, a4paper]{article}
\usepackage{hyperref}
\hypersetup{
  colorlinks=true,
  linkcolor=blue,
  urlcolor=cyan,
}
\urlstyle{same}
\usepackage[utf8]{inputenc}
\usepackage{amsmath}
\usepackage{amsfonts}
\usepackage{amssymb}
\usepackage{graphicx}


\newtheorem{theorem}{Teorema.}
\newtheorem{lemma}[theorem]{Lema.}
\newtheorem{corollary}[theorem]{Corolario.}
\newtheorem{definition}[theorem]{Definici\'on:}
\newtheorem{example}[theorem]{Ejemplo:}
\newtheorem{problema}[theorem]{Problema:}
\newtheorem{remark}[theorem]{Observaci\'on:}

\usepackage{graphicx}
\usepackage[spanish]{babel}
%\usetheme{default}

\newcommand{\pp}{\mathbb{P}}
\newcommand{\zz}{\mathbb{Z}}
\newcommand{\rr}{\mathbb{R}}
\newcommand{\qq}{\mathbb{Q}}

\usepackage{tikz, tikz-3dplot}

\definecolor{cof}{RGB}{219,144,71}
\definecolor{pur}{RGB}{186,146,162}
\definecolor{greeo}{RGB}{91,173,69}
\definecolor{greet}{RGB}{52,111,72}

\date{}

\begin{document}
\title{Pr\'actico 4 ALGABO: Introducción a la programaci\'on dinámica.}
\author{Mauricio Velasco}
\maketitle{}
\begin{enumerate} 



\item Se $G_n$ el grafo de cadena no dirigido con v\'ertices $1,2\dots, n$ (es decir el que tiene aristas $E(G)=\{(j,j+1): j=1,\dots,n-1 \}$ y sus reversas). Suponga que $G$ tiene pesos en los v\'ertices dados por la funci\'on $w: V(G_n)\rightarrow \mathbb{R}$ especificada mediante la fórmula $w(j)=j$ para $j=1,2,\dots, n$ 
\begin{enumerate}
\item Cuál cree usted que es el peso m\'aximo  posible de un subconjunto independiente de vértices de $G_n$? La respuesta debe depender del \'indice $n$.
\item Utilice la formulaci\'on del problema mediante programaci\'on din\'amica para {\it demostrar} que la respuesta que encontraron en la parte $(a)$ es correcta.
\end{enumerate}

\item {\bf M\'aximo conjunto independiente.}  Sea $G$ un grafo no dirigido finito con v\'ertices $v_1,\dots , v_n$ dotado de una funci\'on de pesos $w$ definida en los v\'ertices. Recuerde que $T\subseteq V(G)$ es {\it independiente} si no hay ninguna arista de $G$ que tenga ambos extremos en $T$ y que para un grafo $G$ y un subconjunto $X\subseteq V(G)$ definimos el grafo $G\setminus X$ como aquel que tiene v\'ertices $V(G)\setminus X$ y cuyas aristas son las aristas de $G$ que tienen ambos extremos en $V(G)\setminus X$.
\begin{enumerate}
\item Para un subgrafo $H$ cualquiera de $G$ sea 
\[\Lambda(H)=\max\left\{ w(T): T\subseteq V(H)\text{ y $T$ es independiente}\right\}.\] 
Demuestre que la siguiente ecuaci\'on se cumple 
\[\Lambda(G) = \max\left( \Lambda(G\setminus\{v_n\}), w(v_n) + \Lambda(G\setminus {\rm star}(v_n))\right)\]
donde ${\rm star}(v_n)$ es el conjunto que consiste de $v_n$ y de todos los v\'ertices adyacentes a $v_n$.
\item Proponga un algoritmo de programaci\'on din\'amica para calcular $\Lambda(G)$ usando la ecuaci\'on del rengl\'on anterior.
\item Verdadero \'o Falso? El algoritmo propuesto permite encontrar $\Lambda(G)$ en tiempo lineal para $O(m+n)$ para {\it cualquier} grafo?
\end{enumerate}

\item Considere la siguiente instancia del problema de Knapsack para un saco con capacidad $C=9$
\[
\begin{tabular}{ccc}
Item & Valor & Tama\~no \\
\hline
1 & 1 & 1\\
2 & 2 & 3\\
3 & 3 & 2\\
4 & 4 & 5\\
5 & 5 & 4\\
\end{tabular}
\]
\begin{enumerate}
\item Escriba el c\'odigo en Python de una implementaci\'on de la soluci\'on del problema de Knapsack dada la capacidad, los items y los valores.
\item Usando su programa encuentre el valor \'optimo del problema de arriba y los items que constituyen una soluci\'on de m\'aximo valor. 
\item Escriba la sucesi\'on de subproblemas que su implementaci\'on resuelve con sus respectivos valores \'optimos. 
\end{enumerate}

\item Proponga un algoritmo de programaci\'on din\'amica para resolver el siguiente problema: {\it Dadas capacidades enteras positivas $C_1$ y $C_2$ y una colecci\'on de $n$ items con tama\~nos y capacidades (enteras, positivas) dadas, encuentre dos subconjuntos disjuntos de items $S_1$ y $S_2$ con valor m\'aximo total posible, entre aquellos que pueden meterse en los sacos. Es decir $(S_1,S_2)$ es un maximizador de la funci\'on $W(T_1,T_2):=\sum_{i\in T_1}w(i)+\sum_{i\in T_2}w(i)$ 
entre las parejas $(T_1,T_2)$ de conjuntos de $[n]$ tales que $T_1\cap T_2=\emptyset$ y $\sum_{v\in T_i}\leq C_i$ para $i=1,2$.}

\end{document}

