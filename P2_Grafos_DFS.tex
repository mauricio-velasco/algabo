
\documentclass[12pt, a4paper]{article}
\usepackage{hyperref}
\hypersetup{
  colorlinks=true,
  linkcolor=blue,
  urlcolor=cyan,
}
\urlstyle{same}
\usepackage[utf8]{inputenc}
\usepackage{amsmath}
\usepackage{amsfonts}
\usepackage{amssymb}
\usepackage{graphicx}


\newtheorem{theorem}{Teorema.}
\newtheorem{lemma}[theorem]{Lema.}
\newtheorem{corollary}[theorem]{Corolario.}
\newtheorem{definition}[theorem]{Definici\'on:}
\newtheorem{example}[theorem]{Ejemplo:}
\newtheorem{problema}[theorem]{Problema:}
\newtheorem{remark}[theorem]{Observaci\'on:}

\usepackage{graphicx}
\usepackage[spanish]{babel}
%\usetheme{default}

\newcommand{\pp}{\mathbb{P}}
\newcommand{\zz}{\mathbb{Z}}
\newcommand{\rr}{\mathbb{R}}
\newcommand{\qq}{\mathbb{Q}}

\usepackage{tikz, tikz-3dplot}

\definecolor{cof}{RGB}{219,144,71}
\definecolor{pur}{RGB}{186,146,162}
\definecolor{greeo}{RGB}{91,173,69}
\definecolor{greet}{RGB}{52,111,72}

\date{}

\begin{document}
\title{Pr\'actico 2 ALGABO: Grafos y DFS.}
\author{Mauricio Velasco}
\maketitle{}
\begin{enumerate} 

\item Proponga un algoritmo para listar las componentes conexas de un grafo no dirigido. Demuestre matem\'aticamente que el algoritmo que propone produce la respuesta correcta.

\item Demuestre las siguientes afirmaciones:
\begin{enumerate}
\item En todo grafo dirigido $G$ tenemos $\sum_{v\in V(G)} n_v=|E|$ donde $n_v:=|{\rm Out}(v)|$. C\'omo cambia este enunciado para grafos no dirigidos?
\item Todo par de v\'ertices en un \'arbol esta unido por un \'unico camino de longitud m\'inima.  
\end{enumerate}

\item Visitamos todos los v\'ertices del grafo $G$ no dirigido con $V=\{1,\dots, 7\}$ y con aristas $E$ determinadas por $(1,2), (1,3)$ $(3,7), (3,6), (2,4)$ y $(2,5)$ iniciando en el v\'ertice $(2)$.
\begin{enumerate}
\item Escriba la lista de v\'ertices en el orden en el que los visitar\'iamos en DFS.
\item Hay otro orden posible adicional al que escribi\'o en el numeral anterior?
\item Cu\'antos \'ordenes posibles hay? Escr\'ibalos todos.
\itam Alguno de los anteriores podr\'ia aparecer en una recorrida {\bf BFS} de los v\'ertices de $G$?
\end{enumerate}

\item Haga una implementaci\'on en python de DFS-Topo.
\begin{enumerate}
\item Escriba su c\'odigo de la misma.
\item Qu\'e orden topol\'ogico obtiene para el grafo con v\'ertices $s,v,w,t$ y lista de adyacencia $\{s:[v,w], v:[t], w:[t]\}$?
\item Qu\'e hace el programa si el input es el grafo con v\'ertices $s,v,w$ y lista de adyacencia $\{s:[v], v:[w], w:[s]\}$? Explique su respuesta de manera precisa.
\end{enumerate}

\item Demuestre que un grafo dirigido que admita un orden topol\'ogico $f$ no puede contener un ciclo dirigido.

\end{enumerate}
\end{document}
