
\documentclass[12pt, a4paper]{article}
\usepackage{hyperref}
\hypersetup{
  colorlinks=true,
  linkcolor=blue,
  urlcolor=cyan,
}
\urlstyle{same}
\usepackage[utf8]{inputenc}
\usepackage{amsmath}
\usepackage{amsfonts}
\usepackage{amssymb}
\usepackage{graphicx}


\newtheorem{theorem}{Teorema.}
\newtheorem{lemma}[theorem]{Lema.}
\newtheorem{corollary}[theorem]{Corolario.}
\newtheorem{definition}[theorem]{Definici\'on:}
\newtheorem{example}[theorem]{Ejemplo:}
\newtheorem{problema}[theorem]{Problema:}
\newtheorem{remark}[theorem]{Observaci\'on:}

\usepackage{graphicx}
\usepackage[spanish]{babel}
%\usetheme{default}

\newcommand{\pp}{\mathbb{P}}
\newcommand{\zz}{\mathbb{Z}}
\newcommand{\rr}{\mathbb{R}}
\newcommand{\qq}{\mathbb{Q}}

\usepackage{tikz, tikz-3dplot}

\definecolor{cof}{RGB}{219,144,71}
\definecolor{pur}{RGB}{186,146,162}
\definecolor{greeo}{RGB}{91,173,69}
\definecolor{greet}{RGB}{52,111,72}

\date{}

\begin{document}
\title{Pr\'actico 2 ALGABO: Grafos y DFS.}
\author{Mauricio Velasco}
\maketitle{}
\begin{enumerate} 

\item Proponga un algoritmo para listar las componentes conexas de un grafo no dirigido. Demuestre matem\'aticamente que el algoritmo que propone produce la respuesta correcta.

\item Dibuje un grafo dirigido $G$ cuya versi\'on no-dirigida sea conexo y que tenga tres componentes fuertemente conexas distintas (Dato curioso: Las componentes fuertemente conexas de un grado dirigido se pueden calcular en tiempo lineal usando DFS dos veces mediante el {\it algoritmo de Kosaraju}, proyecto posible?). 


\item Demuestre las siguientes afirmaciones:
\begin{enumerate}
\item En todo grafo dirigido $G$ tenemos $\sum_{v\in V(G)} n_v=|E|$ donde $n_v:=|{\rm Out}(v)|$. C\'omo cambia este enunciado para grafos no dirigidos?
\item Todo par de v\'ertices en un \'arbol esta unido por un \'unico camino de longitud m\'inima.  
\end{enumerate}

\item Visitamos todos los v\'ertices del grafo $G$ no dirigido con $V=\{1,\dots, 7\}$ y con aristas $E$ determinadas por $(1,2), (1,3)$ $(3,7), (3,6), (2,4)$ y $(2,5)$ iniciando en el v\'ertice $(2)$.
\begin{enumerate}
\item Escriba la lista de v\'ertices en el orden en el que los visitar\'iamos en DFS.
\item Hay otro orden posible adicional al que escribi\'o en el numeral anterior?
\item Cu\'antos \'ordenes DFS posibles hay para el grafo? Escr\'ibalos todos.
\end{enumerate}


\item Sea $G$ un grafo finito dirigido.
\begin{enumerate}
\item Complete rigurosamente la siguiente definici\'on: Un orden topol\'ogico para $G$ es una funci\'on... 
\item Demuestre que si $G$ admite un orden topol\'ogico $f$ entonces no puede contener un ciclo dirigido.
\end{enumerate}


\item Haga una implementaci\'on recursiva de \verb TopologicalSort.
\begin{enumerate}
\item Escriba su c\'odigo en Python de la misma.
\item Qu\'e orden topol\'ogico obtiene para el grafo con v\'ertices $s,v,w,t$ y lista de adyacencia $\{s:[v,w], v:[t], w:[t]\}$?
\item Qu\'e hace su implementaci\'on si el input es el grafo con v\'ertices $s,v,w$ y lista de adyacencia $\{s:[v], v:[w], w:[s]\}$? Explique su respuesta de manera precisa.
\end{enumerate}


\end{enumerate}
\end{document}
