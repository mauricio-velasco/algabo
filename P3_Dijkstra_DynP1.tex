
\documentclass[12pt, a4paper]{article}
\usepackage{hyperref}
\hypersetup{
  colorlinks=true,
  linkcolor=blue,
  urlcolor=cyan,
}
\urlstyle{same}
\usepackage[utf8]{inputenc}
\usepackage{amsmath}
\usepackage{amsfonts}
\usepackage{amssymb}
\usepackage{graphicx}


\newtheorem{theorem}{Teorema.}
\newtheorem{lemma}[theorem]{Lema.}
\newtheorem{corollary}[theorem]{Corolario.}
\newtheorem{definition}[theorem]{Definici\'on:}
\newtheorem{example}[theorem]{Ejemplo:}
\newtheorem{problema}[theorem]{Problema:}
\newtheorem{remark}[theorem]{Observaci\'on:}

\usepackage{graphicx}
\usepackage[spanish]{babel}
%\usetheme{default}

\newcommand{\pp}{\mathbb{P}}
\newcommand{\zz}{\mathbb{Z}}
\newcommand{\rr}{\mathbb{R}}
\newcommand{\qq}{\mathbb{Q}}

\usepackage{tikz, tikz-3dplot}

\definecolor{cof}{RGB}{219,144,71}
\definecolor{pur}{RGB}{186,146,162}
\definecolor{greeo}{RGB}{91,173,69}
\definecolor{greet}{RGB}{52,111,72}

\date{}

\begin{document}
\title{Pr\'actico 3 ALGABO: El algoritmo de Dijkstra de caminos m\'as cortos con pesos no negativos.}
\author{Mauricio Velasco}
\maketitle{}
\begin{enumerate} 

\item Sea $G$ un grafo dirigido con pesos positivos $\ell$ en las aristas y sea $t$ un v\'ertice de $G$:
\begin{enumerate}
\item  Proponga una variante del algoritmo de Dijkstra para calcular las distancias {\bf hasta } $t$, es decir los valores $d_{\ell}(v,t)$ para $v\in V(G)$. 
\item Demuestre que su algoritmo es correcto.
\end{enumerate}

\item Implemente el algoritmo de Dijkstra en python dos veces, una vez usando un \verb!priority_queue! o \verb!heap! (por ejemplo usando \verb!import heapq!) y la segunda vez sin usar esta estructura de datos especial.
\begin{enumerate}
\item Escriba el c\'odigo de sus dos implementaciones.
\item Ejecute sus algoritmos en el ejemplo $[M3]$ y escriba en qu\'e orden se incluyen los v\'ertices en el conjunto $X$.
\item Ejecute sus implementaciones en ejemplos construidos por uds. de grafos grandes y lleve sus implementaciones hasta el l\'imite. Es cierto en la pr\'actica que usar el heap es mejor que no usarlo? Explique qu\'e ejemplos de grafos utiliz\'o y escriba una tabla con sus conclusiones.
\end{enumerate}

\item Verdadero \'o Falso? En el primer caso escriba una demostraci\'on y en el segundo encuentre un contrajemplo: {\it Suponga que el algoritmo de Dijkstra iniciando en $s$ inserta los v\'ertices de $V(G)$ en $X$ en orden  $s=v_0,v_1,\dots, v_n$. Si $i<j$ entonces $d_{\ell}(s,v_i)\leq d_{\ell}(s,v_j)$.}


\item Sea $G$ un grafo dirigido con pesos no negativos. Defina el costo-especial de un camino $P$ como 
\[b(P)=\text{M\'axima longitud $\ell$ de una arista de $P$}\]
y para dos v\'ertices $u$ y $v$ defina la distancia-especial $m(s,v)$ como
\[m(s,v)=\min\{b(P): \text{$P$ es un camino desde $s$ hasta $v$ en $G$}\}\]
\begin{enumerate}
\item Calcule $m(s,v)$ para todo $v\in V$ para el grafo $[M3]$ de clase.
\item Proponga una variante del algoritmo de Dijkstra para calcular las distancias especiales $m(s,v)$ para $v\in V(G)$.
\begin{enumerate}
\item Demuestre la validez del algoritmo que propone.
\item Proponga una implementación que corra en tiempo $O(mn)$ justificando de manera precisa su cálculo del tiempo de ejecución.
\end{enumerate}
\end{enumerate}


\item Sea $G$ un \'arbol binario. Demuestre las siguientes afirmaciones:
\begin{enumerate}
\item Si $G$ tiene los primeros $q$ niveles completamente llenos entonces tiene $2^{q}-1$ v\'ertices.
\item Ponemos las entradas de un arreglo de longitud $n$ llenando un \'arbol binario  de izquierda a derecha (note que es posible que el último nivel del \'arbol no quede completamente lleno). Muestre que el número de niveles del \'arbol es $\Theta(\log(n))$ donde $\log$ denota el logaritmo natural (en base $e$) de $n$.
\end{enumerate} 

\end{enumerate}

\end{document}

